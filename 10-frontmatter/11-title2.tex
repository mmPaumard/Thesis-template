\thispagestyle{empty}

\begin{fullwidth}
    \begingroup
       \begin{figure}
            \includegraphics[width=0.4\linewidth]{10-frontmatter/11-su-logo.png}
        \end{figure}
        \vspace{15px}
        
        \begin{center}
            \allcaps{\Large Thèse de Doctorat de Sorbonne Université} \\
            
             \large École doctorale n\kern-0.5bpº\oldstylenums{158} ED3C \\
             
             \small Cerveau, Cognition \textit{\&} Comportement \\[1\baselineskip]
             
             \large Institut du Cerveau --- Équipe Motivation, cerveau \textit{\&} comportement \vfill
        
            {\Huge \textcolor{ceruleanblue}{TODO} \par} \vfill
            {\Huge \adforn{21}} \vfill
            
            Thèse de doctorat de Neurosciences \vfill
            
            Présentée par \\
            {\Large Jules \textsc{Brochard}} \vfill
            
            
            Dirigée par \\
            {\Large Jean \textsc{Daunizeau}} \vfill
            
            \small Présentée et soutenue publiquement le \oldstylenums{TODO} décembre \oldstylenums{2020} \vfill
        \end{center}
        
        \raggedright
        Devant un jury composé de :
        
        \centering
        \begin{tabular}{llr}
    		Jean \textsc{Daunizeau} & Directeur de recherche, Sorbonne Université & Directeur de thèse \\
    	    - \textsc{-} & Professeur des Universités, - & Rapporteur \\
    	    - \textsc{-} & Maître de conférence, - & Rapporteur \\
            - \textsc{-} & Maître de conférence, - & Examinateur \\
    	\end{tabular}

        
        \vspace*{10px}
    \endgroup
\end{fullwidth}

% colophon ============================================
\newpage
\begin{fullwidth}
~\vfill
\thispagestyle{empty}
\setlength{\parindent}{0pt}
\setlength{\parskip}{\baselineskip}
\end{fullwidth}

\paragraph{Colophon}

This document was typeset in \LaTeX, using the beautiful \texttt{tufte-latex} class\footnote[][112ex]{\texttt{tufte-latex} is based on on the work of the famous statistician Edward Tufte.} and a hand-made overlay inspired by Marie-Morgane Paumard's thesis, \textit{Solving jigsaw puzzles with~deep learning}. Firmin Didot's \texttt{GFS Didot} acts as the typeface. The bibliography is typeset using \texttt{biblatex}.


\newthought{}Copyright \copyright\ \the\year\ Author Name

\newthought{}\textit{First printing, Month Year}
